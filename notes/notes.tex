%\documentclass[referee]{aa}
\documentclass{aa}

\usepackage[utf8]{inputenc} 
\usepackage[varg]{txfonts}
\usepackage{amssymb}
\usepackage{epsfig}
\usepackage{graphics}
\usepackage{amsmath}
\usepackage{color}
\usepackage{natbib}
\usepackage{hyperref}

%landscape figures
%\usepackage{wrapfig}
%\usepackage{lscape}
%\usepackage{rotating}

\bibpunct{(}{)}{;}{a}{}{,} % to follow the A&A style

%Debug addition for collaborators
%\usepackage[switch, modulo]{lineno}
%\linenumbers
%\renewcommand\linenumberfont{\color{red}\normalfont\tiny\sffamily}
%\renewcommand\linenumberfont{\normalfont\tiny\sffamily}

\DeclareUnicodeCharacter{00A0}{ }

%Commands
\newcommand{\be}{\begin{equation}}
\newcommand{\ee}{\end{equation}}
\newcommand{\bea}{\begin{align}\begin{split}}
\newcommand{\eea}{\end{split}\end{align}}


\newcommand{\red}[1]{\textcolor{red}{#1}}
\newcommand{\blue}[1]{\textcolor{blue}{#1}}
\newcommand{\green}[1]{\textcolor{green}{#1}}

\newcommand{\Msun}{\ensuremath{\mathrm{M}_{\sun}}}

\newcommand{\D}[2]{\ensuremath{\frac{\partial {#1} }{\partial {#2} }}}
%\newcommand{\f}[1]{\ensuremath{f^{\mathrm{ {#1} }}}}
\newcommand{\f}[1]{\ensuremath{f_{\mathrm{ {#1} }}}}

%normal 3-vectors
%\renewcommand{\vec}[1]{\ensuremath{\mathbf{#1}}}
\renewcommand{\vec}[1]{\ensuremath{\boldsymbol{#1}}}

%four-vectors
\makeatletter
\def\fvec#1{\underline{\sbox\tw@{$#1$}\dp\tw@\z@\box\tw@}}
\makeatother

\newcommand{\dt}{\ensuremath{\Delta t}}

%________________________________________________________________
\begin{document}

\title{Solving Vlasov-Maxwell equations}

\author{J.\,N\"attil\"a\inst{1,2}, A.\,Veledina\inst{2}}

\institute{Tuorla Observatory, Department of Physics and Astronomy, University of Turku, V\"ais\"al\"antie 20, FI-21500 Piikki\"o, Finland \email{joonas.a.nattila@utu.fi}
  \and Nordita, KTH Royal Institute of Technology and Stockholm University, Roslagstullsbacken 23, SE-10691 Stockholm, Sweden
}

\date{Received XXX / Accepted XXX}

\abstract{
Blaa
}


\keywords{numerical}
\titlerunning{PiC notes}

\maketitle

%________________________________________________________________
\section{Introduction}\label{sec:intro}
Blaa blaa!

\section{Code outline}

\subsection{Current minimal working example}
We have implemented a simple minimal working example with following technical details 
\begin{itemize}
    \item Relativistic Vay particle pusher
    \item Yee lattice splitting of $E$ and $B$ fields
    \item trilinear staggered EM-field grid interpolation
    \item Charge conserving current deposit with zigzag-method (Umeda et al)
    \item First-order, cloud-in-the-cell particle shapes
\end{itemize}

\subsection{Technical details}
Technically our implementation has
\begin{itemize}
    \item C++ framework 
    \item Domain decomposition with \textsc{MPI}
    \item Grid communication with non-blocking methods 
    \item Threading with \textsc{openmp}
    \item Adaptive modular grid \textsc{dccrg} (by FMI)
    \item Dynamic load balancing with hypergrap slicing via \textsc{Zoltan}
    \item AVX/SSE vectorization for simple loops
\end{itemize}


\section{Theory}\label{sect:theory}

\subsection{Relativistic Vlasov equation}
Spatial coordinate locations are given by $\vec{x} := (x,y,z)$ and coordinate time is measured with $t$.
From here it follows that the coordinate velocity is given by $\vec{v} := d_t \vec{x}$ and the individual $(x,y,z)$ components of the velocity are denoted as $\vec{v} = (v_x, v_y, v_z)$.
Proper time (measured with a co-moving clock) is $\tau$ and it is connected to the coordinate time with the Lorentz factor $\gamma := d_{\tau} t$.
Similarly, the proper velocity is then $\vec{u} := d_{\tau} \vec{x} = \gamma \vec{v}$.
Lorentz factor and the (proper) velocity are connected by the expression
\be
\gamma^2 = 1 + (u/c)^2 = (1-(v/c)^2)^{-1},
\ee
where $u = |\vec{u}|$ and $v = |\vec{v}|$.
Acceleration is denoted with $\vec{a} := d_t \vec{u}$.

Six dimensional phase-space density distribution for species particle s is given by $\f{s} := \f{s}(t, \vec{x}, \vec{u})$.
Evolution of $\f{s}$ is governed by the relativistic Vlasov equation
\be\label{eq:vlasov}
\D{\f{s}}{t} + \nabla_{\vec{x}} \cdot (\vec{v} \f{s}) + \nabla_{\vec{u}} \cdot (\vec{a}_{\mathrm{s}} \f{s} ) = \mathcal{C},
\ee
where $\mathcal{C} := \partial_t \f{s} ~|_{\mathrm{coll}}$ is the collision operator.
When studying plasma, we can provide a closure for the acceleration with the Lorentz force as 
\be
\vec{a}_{\mathrm{s}} = d_t \vec{u} = (q_{\mathrm{s}} / m_{\mathrm{s}} )(\vec{E} + \vec{v} \times \vec{B}),
\ee
where $\vec{E}$ and $\vec{B}$ are the electric and magnetic fields, $q_{\mathrm{s}}$ is the charge of particles in species s, and $m_{\mathrm{s}}$ is the particle mass.
First moment of the distribution function $\f{s}$ gives us the charge density
\be
\rho := \sum_{\mathrm{s}} q_{\mathrm{s}} \int \f{s} d\vec{u} = \sum_{\mathrm{s}} q_{\mathrm{s}} \int \f{s} \, \gamma \, \frac{ ~d\vec{u}}{\gamma}.
\ee
Similarly, the second moment defines the current
\be\label{eq:J}
\vec{J} := \sum_{\mathrm{s}} q_{\mathrm{s}} \int \f{s} \vec{v} d\vec{u} = \sum_{\mathrm{s}} q_{\mathrm{s}} \int \f{s} \, \vec{u} ~\frac{ d\vec{u}}{\gamma}.
\ee

To solve \eqref{eq:vlasov} numerically we will use first-order splitting by Cheng \& Knorr to decouple spatial and momentum space as
\be
\D{\f{s}}{t} + \nabla_{\vec{x}} \cdot (\vec{v} \f{s} ) = 0
\ee
\be
\D{\f{s}}{t} +\frac{q_{\mathrm{s}}}{m_{\mathrm{s}}} \nabla_{\vec{a}} \cdot ( \vec{E} + \vec{v} \times \vec{B} ) \f{s} = 0
\ee
\be
\D{\f{s}}{t} = \mathcal{C}
\ee
Conservative update in one dimension (labeld with $i$ and $x$) is then applied as
\be
f_{i}^{t+\Delta t} = f_{i}^{t} + \frac{q_{\mathrm{s}}}{m_{\mathrm{s}}}  \frac{\Delta x}{\Delta t}( U_{x, i+\frac{1}{2}}^t - U_{x, i-\frac{1}{2}}^t ),
\ee
from which we also get the charge conserved current \eqref{eq:J}
\be
J_x = \sum_{\mathrm{s}} U_{x, i+\frac{1}{2}}.
\ee



\subsection{General relativistic Maxwell equations}
The electric field $E$ and magnetic field $B$ are described by the Maxwell's equations.
Shape of the fields are given by the curls
\be
\nabla \times \vec{E} = \frac{1}{c} \frac{\partial \vec{B}} {\partial t}
\ee
\be
\nabla \times \vec{B} = \frac{4 \pi \vec{J}}{c} + \frac{1}{c} \frac{\partial \vec{E}} {\partial t}
\ee
whereas initial conditions come from
\be
\nabla \cdot \vec{E} = 4\pi \rho,
\ee
\be
\nabla \cdot \vec{B} = 0.
\ee
As we will later on see, a charge conserving current deposition and Yee lattice staggering of the fields will take care of the last two equations.
Then we are left with updating the fields from the curls.
We solve
\be
\frac{\partial \vec{E}}{\partial t} = c [ \nabla \times \vec{B}] - 4\pi \vec{J}
\ee
\be
\frac{\partial \vec{B}}{\partial t} = -c[ \nabla \times \vec{E}]
\ee

Next we can cast this into $3+1$ formalism following \citealt{P15}.
For this, we split the four dimensional space-time into a foliation where 
\be
ds^2 = \alpha^2 c^2 dt^2 - \gamma_{ab}(dx^a + \beta^a c~dt)(dx^b + \beta^b c~dt),
\ee
where $x^i = (c~t, x^a)$, $t$ is the time coordinate or the universal time and $x^a$ some associated space coordinate.
Here the general relativistic corrections are introduced via the lapse function $\alpha$, shift vector $\beta^a$, and spatial metric of absolute space $\gamma_{ab}$.
By convention Latin letters from $a$ to $h$ are used for components of vectors in absolute space (in range ${1,2,3}$) whereas Latin letters starting from $i$ correspond to the four-dimensional vectors and tensors (${0,1,2,3}$).
Note that Landau-Lifschitz signature of $(+,-,-,-)$ is used.

In some dispersive media we have
\be
\epsilon_0 \vec{E} = \vec{\alpha} \vec{D} + \epsilon_0 c \beta \times \vec{B}
\ee
\be
\mu_0 \vec{H} = \vec{\alpha} \vec{B} - \frac{\beta \times \vec{D}}{\epsilon_0 c}
\ee
where $\epsilon_0$ is the vacuum permettivity and $\mu_0$ is the vacuum permeability.
Auxiliary vector fields denoted by $\vec{F}$ and $\vec{G}$ can be introduced as
\be
\vec{F} = \zeta_1 \vec{D} + \frac{\zeta_2}{c} \vec{B}
\ee
\be
\vec{G} = \zeta_1 \vec{H} - \frac{\zeta_2}{c} \vec{E}.
\ee

The inhomogeneous Maxwell equations are
\be
\nabla \cdot \vec{F} = \rho
\ee
\be
\nabla \times \vec{G} = \vec{J} + \frac{1}{\sqrt{\gamma}} \partial_t (\sqrt{\gamma} \vec{F}),
\ee
where xxx

The homogeneous equations are
\be
\nabla \times \vec{E} = -\frac{1}{\sqrt{\gamma}} \partial_t (\sqrt{\gamma} \vec{B})\ee
\be
\nabla \cdot \vec{B} = 0.
\ee

When neglecting QED corrections we can set $\zeta_1=1$ and $\zeta_2=0$.
Similarly, when working in vacuum $\vec{D}$ and $\vec{H}$ reduce to normal $\vec{E}$ and $\vec{B}$.

\subsection{Nondimensionalization}
Let us next make our set of equations non-dimensional to help numerical calculations.
Physical constants that define an ion-electron plasma are the magnitude of an electron charge $e$, the ion and electron masses $m_{\mathrm{i}}$ and $m_{\mathrm{e}}$, and the speed of light $c$.
Additionally, three fundamental parameters charaterize the plasma states: a typical particle density $n_0$, a typical temperature $T_0$ (per species s), and a typical magnetic field strength $B_0$.
For a quasineutral plasma we have $n_0 = n_{\mathrm{i}} = n_{\mathrm{e}}$ and $T_0 = T_{\mathrm{i}} = T_{\mathrm{e}}$.
Thermal pressure is $P_0 := n_0 T_0$ and magnetic pressure is $P_B := \frac{B_0^2}{2 \mu_0}$.

Auxiliary space, time and velocity scale parameters derived from these fundamental parameters are the gyrofrequency
\be
\omega_{\mathrm{g,s}} = \frac{e B_0}{m_{\mathrm{s}}},
\ee
the plasma frequency
\be
\omega_{\mathrm{p,s}} = \sqrt{ \frac{n_0 e^2}{\epsilon_0 m_{\mathrm{s}}}},
\ee
thermal velocity
\be
v_{\mathrm{T,s}} = \sqrt{ \frac{T_{\mathrm{s}} }{m_{\mathrm{s}}} },
\ee
for which the average velocity is easily defined as $\tilde{v}_{\mathrm{T,s}}^2 = 2 v_{\mathrm{T,s}}^2$, and 
Alfvén velocity
\be
v_{\mathrm{A, s}} = \sqrt{ \frac{B_0^2}{\mu_0 m_{\mathrm{s}} n_0} }.
\ee
From a combination of the above we obtain the gyroradii
\be
r_{\mathrm{g,s}} = \frac{v_{\mathrm{T,s}} }{ \omega_{\mathrm{g, s}} }  = \frac{ m_{\mathrm{s}} v_{\mathrm{T,s}} }{e B_0},
\ee
the Debye length
\be
\lambda_{\mathrm{D}} = \frac{ v_{\mathrm{T,s}} }{\omega_{\mathrm{p,s}}} = \sqrt{ \frac{ \epsilon_0 T_0 }{n_0 e^2} },
\ee
and the skin depth
\be
\delta_{\mathrm{s}} = \frac{c}{\omega_{\mathrm{p,s}} } = \frac{v_{\mathrm{A,s}} }{\omega_{\mathrm{g,s}}} = \sqrt{ \frac{m_{\mathrm{s}}}{\mu_0 n_{\mathrm{s}} e^2}}.
\ee
Important nondimensional ratios are the plasma beta $\beta := \frac{P_0}{P_B}$ and the ratio of the speed of light to the the Alvén speed.
These two imply
\be
\beta = \frac{P_0}{P_B} = \left( \frac{\tilde{v}_{\mathrm{T,s}} }{v_{\mathrm{A,s}} } \right)^2 = \left( \frac{\tilde{r}_{\mathrm{g,s}} }{\delta_{\mathrm{s}} } \right)^2,
\ee
and
\be
\frac{c}{v_{\mathrm{A,s}}} = \frac{r_{\mathrm{g,s}} }{\lambda_{\mathrm{D}} } = \frac{\omega_{\mathrm{p,s}}}{\omega_{\mathrm{g,s}}}.
\ee


These give us the almost identical nondimensionalized system
\begin{align}\begin{split}
    \partial_{\tilde{t}} \tilde{\vec{B}} &= -\nabla_{\tilde{x}} \tilde{\vec{E}}, \\
    \partial_{\tilde{t}} \tilde{\vec{E}} &=  \tilde{c}^2 \, \nabla_{\tilde{x}} \tilde{\vec{B}} - \frac{\tilde{\vec{J}}}{ \red{\epsilon} } \\ 
    \nabla_{\tilde{\vec{x}}} \cdot \tilde{\vec{B}} &= 0 \\
    \nabla_{\tilde{\vec{x}}} \cdot \tilde{\vec{E}} &= \frac{ \tilde{\rho} }{\red{\epsilon}} \\
    \tilde{\rho} &= \sum_{\mathrm{s}} \tilde{S}(\tilde{\vec{x}}) \tilde{q}_{\mathrm{s}} \\
    \tilde{\vec{J}} &= \sum_{\mathrm{s}} \tilde{S}(\tilde{\vec{x}}) \tilde{q}_{\mathrm{s}} \tilde{\vec{v}} \\
    d_{\tilde{t}}(\gamma \tilde{\vec{v}} ) &= \red{(t_0 \omega_{\mathrm{g}})} \frac{\tilde{q}_{\mathrm{s}}}{\tilde{m_{\mathrm{s}}}} \left( \tilde{\vec{E}}(\tilde{\vec{x}}) + \tilde{\vec{v}} \times \tilde{\vec{B}}(\tilde{\vec{x}}) \right) \\
d_{\tilde{t}} \tilde{\vec{x}} &= \tilde{\vec{v}} \\
\end{split}\end{align}

\be
t_0 \omega_{\mathrm{g}} = t_0 \frac{q_0 B_0}{m_0}
\ee
and
\be
\frac{1}{\epsilon} = \frac{x_0 n_0 e}{v_0 B_0 \epsilon_0} = t_0 \frac{e B_0}{m_0} \frac{\mu_0 m_0 n_0}{B_0^2} = (t_0 \omega_{\mathrm{g}})\left( \frac{c}{v_{\mathrm{A}}} \right)^2.
\ee

\subsection{Particle-in-Cell method}

Collisionality of a plasma comprising of point particles is modeled by finite-sized particle clouds (or macro particles) with radius $\epsilon$.
Typical cloud size is usually orders of magnitude larger compared to real plasma so that number of particles in Debye sphere $N_{\mathrm{D}} = \frac{4 \pi}{3}n\lambda_{\mathrm{D}}^3$ is effectively replaced by the parameter $N_{\mathrm{C}} = \frac{4 \pi}{3}n\epsilon^3$, where $n$, and $\lambda_{\mathrm{D}}$ are the number density and Debye length, respectively.
In PIC codes the particle size is typically equivalent to grid spacing $\Delta$, which must be kept $\le \lambda_{\mathrm{d}}$ to avoid aliasing instabilities that usually manifest themselves as numerical heating.
To map the particle densities smoothly into to the mesh it is also favorable to have as many particles as possible in the simulations.

\subsection{Field updates}
In finite difference terms one obtains 
\be
\vec{E}^{n+1/2} = \vec{E}^{n-1/2} + \dt [c(\nabla \times \vec{E}) - 4\pi \vec{J}^n]
\ee
and
\be
\vec{B}^{n+1} = \vec{B}^n - c \dt [\nabla \times \vec{E}].
\ee

Curls can be computed from Stokes theorem
\be
\iint \nabla \times \vec{E} ~d\vec{S} = \oint \vec{E} \cdot d\vec{l}
\ee
that in finite differencing scheme reduces to 
\begin{align}\begin{split}
\iint_{S}& \left[  \left( \frac{\partial E_z}{\partial y} - \frac{\partial E_x}{\partial x} \right) \vec{\hat{i}} + 
\left( \frac{\partial E_x}{\partial z} - \frac{\partial E_z}{\partial x} \right) \vec{\hat{j}} + 
\left( \frac{\partial E_y}{\partial x} - \frac{\partial E_x}{\partial y} \right) \vec{\hat{k}} \right] ~d\vec{S} \\
&= \oint_{\partial S} E_x dx \vec{\hat{i}} +  E_y dy \vec{\hat{j}} + E_z dz \vec{\hat{k}} \\
\end{split}\end{align}



\subsection{Relativistic Vay pusher}


We start with velocities at a time $i$ given as $u = \gamma v$. 
In addition we can compactify the notation by introducing $h = \frac{q \Delta t}{2 m c}$.
Then we update the first half as
\begin{enumerate}
    \item $\gamma^i = \sqrt{1 + {u^i)^2}}$
    \item $u^{i+1/2} = u^i + \frac{q \Delta t}{2 m c} (E^{i+1/2} + v^i \times B^{i+1/2}) = u^i + h (E^{i+1/2} + v^i \times B^{i + 1/2})$
\end{enumerate}
Second half is done by computing the auxiliary steps $1-8$ and combining them in step $9$ as
\begin{enumerate}
    \item $u' = u^{i+1/2} + \frac{q \Delta t}{2 m c} E^{i+1/2} = u^{i+1/2} + h E^{i+1/2}$
    \item $\tau = \frac{q \Delta t}{2 m c} B^{i+1/2} = h B^{i + 1/2}$
    \item $\gamma' = \sqrt{1 + {u'}^2}$
    \item $\sigma = {\gamma'}^2 - \tau^2$
    \item $u^{*} = u' \cdot \tau^2$
    \item $\gamma^{i+1} = \sqrt{ \frac{\sigma + \sqrt{\sigma^2 + 4(\tau^2 + {u^{*}}^2)}}{2}}$
    \item $t = \tau / \gamma^{i+1}$
    \item $s = \frac{1}{1+(\tau/\gamma^{i+1})^2}$
    \item $u^{i+1} = s[u' + (u' \cdot t) t + u' \times t]$
\end{enumerate}

\subsection{Radiation reaction}

The radiation reaction (RR) force appears as a friction term with a nonlinear and anisotropic friction coefficient. 
When electron crosses an EM field it feels a viscous force opposite to its velocity.

The friction effect of the RR physically corresponds to the incoherent emission of high-frequency radiation by ultra-relativistic electrons.
When the RR is included, it is typically not feasible to resolve electromagnetic waves at such high frequencies.
Thus, it is assumed that such radiation escapes from the system without re-interacting with other electrons.
From the point of view of the energy balance then, the energy radiated at high frequencies appear as a loss term or dissipation.

Hence, the force must be added because such a high frequency is not resolved by the grid, and hence the back reaction can not be captured.


In the conditions typical for the accretion flows in the hard-state X-ray binaries, the assumption of optically thin plasma breaks down.
Radiation cannot freely escape from the system, and is rather playing an important role in particle thermalization processes.
The synchrotron boiler [Ghisellini et al. 1988, 1997; Poutanen \& Vurm 2009; Malzac \& Belmont 2009] proved to be very efficient mechanism for cooling 
electrons in hot accretion flows of both Galactic X-ray binaries and Seyferts.

We choose to incorporate the radiation reaction in the plasma simulations using the formalism of distribution functions.
General form of kinetic equation for particles and photons is given by equation~\ref{eq:vlasov}, where we now define the collisional operator for particles as 
due to synchrotron processes.
Its direct form is however not easy to determine (compared to the case of binary collisions, such as Compton scattering or pair processes), as the interaction 
between the electron and emitted/absorbed photon takes place through the magnetic field, which plays a role of an external force (in fact, the collisional operator 
is derived under the assumption of binary collision, which breaks down in our case).
The total energy of the electron-photon system is conserved, however the three-momentum is not, and it is compensated for by the field.
Instead, the equation for synchrotron processes is usually derived employing 





\section{Possible projects and set up}

\subsection{Synchrotron boiler from the first principles}

Phenomenological models: Ghisellini et al. 1988, 1997; Poutanen \& Vurm 2009; Malzac \& Belmont 2009


\subsection{MRI and particle energization in accretion flows around AGN} 

The X-ray spectra of Seyfert 1 galaxies can be modelled with Comptonization continuum with Thomson optical depth 
$\tau \sim 1$, from which we deduce the number density of electrons $n_{\rm e}\sim 10^{11}$~cm$^{-3}$, relevant 
length-scales where the energy is liberated are about $10R_{\rm S}=2.95\times10^{13}$~cm.
The electron temperatures are about 100~keV, similar to those found in black hole binaries.
We assume that the magnetic energy density is of the order of the radiation energy density, say, $U_{\rm B}=U_{\rm rad}/3$. 
For $L=10^{-2}L_{\rm Edd}\sim10^{39}M/M_{\odot}$~erg~s$^{-1}$ and for $M=10^7M_{\odot}$ we have 
$B^2 = \frac {1}{3} 8\pi \frac {L}{(4\pi/3) R^2 c}\sim 10^6$, so $B=10^3$~G.

From these we get the relevant time- and length-scales:
\begin{itemize}
\item Electron plasma frequency 
$\omega_{\rm pe} = \sqrt{ \frac{4\pi n_{\rm e} e^2}{m_{\rm e}} } 
                 = 5.64 \times 10^4 \sqrt{n_{\rm e}}=1.78\times10^{10}$~rad~s${-1}$
\item Electron Larmor frequency (for electron Lorentz factor $\gamma=1$)
$\omega_{\rm ce} = \frac {eB}{m_{\rm e}c} = 1.76 \times 10^7 B = 1.76\times10^{10}$~~rad~s${-1}$
\item Electron skin depth
$\lambda_{\rm skin,e} = \frac {c}{\omega_{\rm pe}} = 5.31\times10^5 n_{\rm e}^{-1/2} = 1.7$~cm
\item Debye radius
$\lambda_{\rm D} = \sqrt{ \frac {k T_{\rm e}}{4\pi n_{\rm e} e^2} } 
                 = 7.43\times10^2 \sqrt{ \frac {T_{\rm e} {\rm[eV]}}{n_{\rm e}}} = 0.74$~cm
\item Mean free path for electrons ($T_{\rm e}$ in eV)
$\lambda_{\rm free,e} = \frac {\overline{v}}{\nu_{\rm ee}} \sim 1.43\times 10^{13} \frac {T_{\rm e}^2}{n_{\rm e} ln\Lambda} 
                      \sim 10^{11}$~cm.
\end{itemize}
So the system is collisionless.



Relevant references: 
\begin{enumerate}
\item \href{http://adsabs.harvard.edu/abs/2016arXiv160807911K}{Kunz et al. 2016} -- found that the angular momentum transport  is enhanced and the particles (protons) acquire non-thermal distribution in the collisionless system
\item \href{http://adsabs.harvard.edu/abs/2015ApJ...800...89S}{Sironi 2015} 
and \href{http://adsabs.harvard.edu/abs/2015ApJ...800...88S}{Sironi \& Narayan 2015} -- electron heating (acceleration) mechanisms in two-temperature collisionless compressing flows.
\end{enumerate}


\subsection{Radiative magnetic reconnection in accretion flows}

Relevant references:
\begin{enumerate}
\item \href{http://adsabs.harvard.edu/abs/2017arXiv170102847B}{Beloborodov 2017} -- analytically estimated radiative effects on particle energization.
\item \href{http://adsabs.harvard.edu/abs/2014ApJ...783L..21S}{Sironi \& Spitkovsky 2014} -- PiC simulations of relativistic reconnection in pair plasma
\item \href{http://adsabs.harvard.edu/abs/2016MNRAS.462...48S}{Sironi et al 2016} -- behaviour of plasmoids in relativistic reconnection 
\item \href{http://adsabs.harvard.edu/abs/2016arXiv161204493W}{Werner et al. 2016} -- electron-ion relativistic reconnection
\end{enumerate}

\section*{Acknowledgments}
This research was supported by the V\"ais\"al\"a Foundation and by the University of Turku Graduate School in Physical and Chemical Sciences (JN).
This research was undertaken on Finnish Grid Infrastructure (FGI) resources.


\bibliographystyle{aa}
\bibliography{allbib}

%\begin{thebibliography}{62}
%\end{thebibliography}



\clearpage
%\begin{appendix}
%\section{Appendix}
%
%\end{appendix}






\end{document}
