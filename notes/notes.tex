%\documentclass[referee]{aa}
\documentclass{aa}

\usepackage[utf8]{inputenc} 
\usepackage[varg]{txfonts}
\usepackage{amssymb}
\usepackage{epsfig}
\usepackage{graphics}
\usepackage{amsmath}
\usepackage{color}
\usepackage{natbib}
\usepackage{hyperref}

%landscape figures
%\usepackage{wrapfig}
%\usepackage{lscape}
%\usepackage{rotating}

\bibpunct{(}{)}{;}{a}{}{,} % to follow the A&A style

%Debug addition for collaborators
%\usepackage[switch, modulo]{lineno}
%\linenumbers
%\renewcommand\linenumberfont{\color{red}\normalfont\tiny\sffamily}
%\renewcommand\linenumberfont{\normalfont\tiny\sffamily}

\DeclareUnicodeCharacter{00A0}{ }

%Commands
\newcommand{\be}{\begin{equation}}
\newcommand{\ee}{\end{equation}}
\newcommand{\bea}{\begin{align}\begin{split}}
\newcommand{\eea}{\end{split}\end{align}}


\newcommand{\red}[1]{\textcolor{red}{#1}}
\newcommand{\blue}[1]{\textcolor{blue}{#1}}
\newcommand{\green}[1]{\textcolor{green}{#1}}

\newcommand{\Msun}{\ensuremath{\mathrm{M}_{\sun}}}


%normal 3-vectors
%\renewcommand{\vec}[1]{\ensuremath{\mathbf{#1}}}
\renewcommand{\vec}[1]{\ensuremath{\boldsymbol{#1}}}

%four-vectors
\makeatletter
\def\fvec#1{\underline{\sbox\tw@{$#1$}\dp\tw@\z@\box\tw@}}
\makeatother

\newcommand{\dt}{\ensuremath{\Delta t}}

%________________________________________________________________
\begin{document}

\title{Solving Vlasov-Maxwell equations with PiC}

\author{J.\,N\"attil\"a\inst{1,2} }

\institute{Tuorla Observatory, Department of Physics and Astronomy, University of Turku, V\"ais\"al\"antie 20, FI-21500 Piikki\"o, Finland \email{joonas.a.nattila@utu.fi}
  \and Nordita, KTH Royal Institute of Technology and Stockholm University, Roslagstullsbacken 23, SE-10691 Stockholm, Sweden
}

\date{Received XXX / Accepted XXX}

\abstract{
Blaa
}


\keywords{numerical}
\titlerunning{PiC notes}

\maketitle

%________________________________________________________________
\section{Introduction}\label{sec:intro}
Blaa blaa!

\section{Theory}\label{sect:theory}

Solving Vlasov-Maxwell equations.


\subsection{Particle-in-Cell method}

Collisionality of a plasma comprising of point particles is modeled by finite-sized particle clouds (or macro particles) with radius $\epsilon$.
Typical cloud size is usually orders of magnitude larger compared to real plasma so that number of particles in Debye sphere $N_{\mathrm{D}} = \frac{4 \pi}{3}n\lambda_{\mathrm{D}}^3$ is effectively replaced by the parameter $N_{\mathrm{C}} = \frac{4 \pi}{3}n\epsilon^3$, where $n$, and $\lambda_{\mathrm{D}}$ are the number density and Debye length, respectively.
In PIC codes the particle size is typically equivalent to grid spacing $\Delta$, which must be kept $\le \lambda_{\mathrm{d}}$ to avoid aliasing instabilities that usually manifest themselves as numerical heating.
To map the particle densities smoothly into to the mesh it is also favorable to have as many particles as possible in the simulations.



\subsection{Maxwell's equations}

Shape of the fields are given by the curls
\be
\nabla \times \vec{E} = \frac{1}{c} \frac{\partial \vec{B}} {\partial t}
\ee
\be
\nabla \times \vec{B} = \frac{4 \pi \vec{J}}{c} + \frac{1}{c} \frac{\partial \vec{E}} {\partial t}
\ee
whereas initial conditions from
\be
\nabla \cdot \vec{E} = 4\pi\vec{J},
\ee
\be
\nabla \cdot \vec{B} = 0,
\ee
which are taken care by charge conserving particle mover and Yee lattice configuration.

From here we solve
\be
\frac{\partial \vec{E}}{\partial t} = c [ \nabla \times \vec{B}] - 4\pi \vec{J}
\ee

\be
\frac{\partial \vec{B}}{\partial t} = -c[ \nabla \times \vec{E}]
\ee

And in finite difference terms one obtains 
\be
\vec{B}^{n+1} = \vec{B}^n - \dt c [\nabla \times \vec{E}]
\ee

\be
\vec{E}^{n+1/2} = \vec{E}^{n-1/2} + \dt [c(\nabla \times \vec{E}) - 4\pi \vec{J}^n]
\ee

Curls can be computed from Stokes theorem
\be
\iint \nabla \times \vec{E} d\vec{S} = \oint \vec{E} \cdot d\vec{l}
\ee
that in finite differencing scheme reduces to 
\begin{align}\begin{split}
\iint_{S}& \left[  \left( \frac{\partial E_z}{\partial y} - \frac{\partial E_x}{\partial x} \right) \vec{\hat{i}} + 
\left( \frac{\partial E_x}{\partial z} - \frac{\partial E_z}{\partial x} \right) \vec{\hat{j}} + 
\left( \frac{\partial E_y}{\partial x} - \frac{\partial E_x}{\partial y} \right) \vec{\hat{k}} \right] ~d\vec{S} \\
&= \oint_{\partial S} E_x dx \vec{\hat{i}} +  E_y dy \vec{\hat{j}} + E_z dz \vec{\hat{k}} \\
\end{split}\end{align}

\subsection{General relativistic Maxwell equations}
Next we can cast this into $3+1$ formalism following \citealt{P15}.
For this, we split the four dimensional space-time into a foliation where 
\be
ds^2 = \alpha^2 c^2 dt^2 - \gamma_{ab}(dx^a + \beta^a c~dt)(dx^b + \beta^b c~dt),
\ee
where $x^i = (c~t, x^a)$, $t$ is the time coordinate or the universal time and $x^a$ some associated space coordinate.
Here the general relativistic corrections are introduced via the lapse function $\alpha$, shift vector $\beta^a$, and spatial metric of absolute space $\gamma_{ab}$.
By convention Latin letters from $a$ to $h$ are used for components of vectors in absolute space (in range ${1,2,3}$) whereas Latin letters starting from $i$ correspond to the four-dimensional vectors and tensors (${0,1,2,3}$).
Note that Landau-Lifschitz signature of $(+,-,-,-)$ is used.

In some dispersive media we, on the other hand, have
\be
\epsilon_0 \vec{E} = \vec{\alpha} \vec{D} + \epsilon_0 c \beta \times \vec{B}
\ee
\be
\mu_0 \vec{H} = \vec{\alpha} \vec{B} - \frac{\beta \times \vec{D}}{\epsilon_0 c}
\ee
where $\epsilon_0$ is the vacuum permettivity and $\mu_0$ is the vacuum permeability.
Auxiliary vector fields denoted by $\vec{F}$ and $\vec{G}$ can be introduced as
\be
\vec{F} = \zeta_1 \vec{D} + \frac{\zeta_2}{c} \vec{B}
\ee
\be
\vec{G} = \zeta_1 \vec{H} - \frac{\zeta_2}{c} \vec{E}.
\ee

The inhomogeneous Maxwell equations are
\be
\nabla \cdot \vec{F} = \rho
\ee
\be
\nabla \times \vec{G} = \vec{J} + \frac{1}{\sqrt{\gamma}} \partial_t (\sqrt{\gamma} \vec{F}),
\ee
where xxx

The homogeneous equations are
\be
\nabla \times \vec{E} = -\frac{1}{\sqrt{\gamma}} \partial_t (\sqrt{\gamma} \vec{B})\ee
\be
\nabla \cdot \vec{B} = 0.
\ee

Hence, when neglecting QED corrections we can set $\zeta_1=1$ and $\zeta_2=0$.
Similarly, when working in vacuum $\vec{D}$ and $\vec{H}$ reduce to normal $\vec{E}$ and $\vec{B}$.



\subsection{Radiation reaction}

The radiation reaction (RR) force appears as a friction term with a nonlinear and anisotropic friction coefficient. 
When electron crosses an EM field it feels a viscous force opposite to its velocity.

The friction effect of the RR physically corresponds to the incoherent emission of high-frequency radiation by ultra-relativistic electrons.
When the RR is included, it is typically not feasible to resolve electromagnetic waves at such high frequencies.
Thus, it is assumed that such radiation escapes from the system without re-interacting with other electrons.
From the point of view of the energy balance then, the energy radiated at high frequencies appear as a loss term or dissipation.

Hence, the force must be added because such a high frequency is not resolved by the grid, and hence the back reaction can not be captured.




\section{Possible projects and set up}

\subsection{MRI and particle energization in accretion flows around AGN} 

The X-ray spectra of Seyfert 1 galaxies can be modelled with Comptonization continuum with Thomson optical depth 
$\tau \sim 1$, from which we deduce the number density of electrons $n_{\rm e}\sim 10^{11}$~cm$^{-3}$, relevant 
length-scales where the energy is liberated are about $10R_{\rm S}=2.95\times10^{13}$~cm.
The electron temperatures are about 100~keV, similar to those found in black hole binaries.
We assume that the magnetic energy density is of the order of the radiation energy density, say, $U_{\rm B}=U_{\rm rad}/3$. 
For $L=10^{-2}L_{\rm Edd}\sim10^{39}M/M_{\odot}$~erg~s$^{-1}$ and for $M=10^7M_{\odot}$ we have 
$B^2 = \frac {1}{3} 8\pi \frac {L}{(4\pi/3) R^2 c}\sim 10^6$, so $B=10^3$~G.

From these we get the relevant time- and length-scales:
\begin{itemize}
\item Electron plasma frequency 
$\omega_{\rm pe} = \sqrt{ \frac{4\pi n_{\rm e} e^2}{m_{\rm e}} } 
                 = 5.64 \times 10^4 \sqrt{n_{\rm e}}=1.78\times10^{10}$~rad~s${-1}$
\item Electron Larmor frequency (for electron Lorentz factor $\gamma=1$)
$\omega_{\rm ce} = \frac {eB}{m_{\rm e}c} = 1.76 \times 10^7 B = 1.76\times10^{10}$~~rad~s${-1}$
\item Electron skin depth
$\lambda_{\rm skin,e} = \frac {c}{\omega_{\rm pe}} = 5.31\times10^5 n_{\rm e}^{-1/2} = 1.7$~cm
\item Debye radius
$\lambda_{\rm D} = \sqrt{ \frac {k T_{\rm e}}{4\pi n_{\rm e} e^2} } 
                 = 7.43\times10^2 \sqrt{ \frac {T_{\rm e} {\rm[eV]}}{n_{\rm e}}} = 0.74$~cm
\item Mean free path for electrons ($T_{\rm e}$ in eV)
$\lambda_{\rm free,e} = \frac {\overline{v}}{\nu_{\rm ee}} \sim 1.43\times 10^{13} \frac {T_{\rm e}^2}{n_{\rm e} ln\Lambda} 
                      \sim 10^{11}$~cm.
\end{itemize}
So the system is collisionless.



Relevant references: 
\begin{enumerate}
\item \href{http://adsabs.harvard.edu/abs/2016arXiv160807911K}{Kunz et al. 2016} - found that the angular momentum transport  is enchanced and the particles (protons) aquire non-thermal distribution in the collisionless system
\item \href{http://adsabs.harvard.edu/abs/2015ApJ...800...89S}{Sironi 2015} 
and \href{http://adsabs.harvard.edu/abs/2015ApJ...800...88S}{Sironi \& Narayan 2015} - electron heating (acceleration) mechanisms in two-temperature collisionless compressing flows.
\end{enumerate}



\section*{Acknowledgments}
This research was supported by the V\"ais\"al\"a Foundation and by the University of Turku Graduate School in Physical and Chemical Sciences (JN).
This research was undertaken on Finnish Grid Infrastructure (FGI) resources.


\bibliographystyle{aa}
\bibliography{allbib}

%\begin{thebibliography}{62}
%\end{thebibliography}



\clearpage
\begin{appendix}
\section{Appendix}


\end{appendix}
\end{document}
